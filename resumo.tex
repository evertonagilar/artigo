\begin{abstract}
A modernização dos sistemas legados é um processo que vem ganhando 
cada vez mais interesse na Universidade de Brasília (UnB). 
Entre os desafios envolvidos na condução da modernização na UnB, 
pelo seu Centro de Informática (CPD), destaca-se a ausência de 
integração entre as aplicações e as duplicidades de componentes que implementam lógica de negócio. Assim, 
é imprescindível que, enquanto a modernização seja realizada, os novos sistemas sejam integrados aos antigos, de forma a 
interagir e compartilhar os seus fluxos de negócios. A Arquitetura Orientada a Serviços (SOA) surge como uma maneira de solucionar 
este problema, disponibilizando uma abstração de alto nível entre as aplicações e a camada de negócio.
Este artigo descreve os resultados preliminares de uma 
dissertação de mestrado que objetiva propor e validar uma 
abordagem orientada a serviços que compreende um processo 
de moderniza\c c\~{a}o e um barramento aderente ao estilo 
arquitetural \textit{Representational State Transfer} (REST). 
Esta abordagem visa sustentar a integração dos fluxos de informações e minimizar 
as duplicidades de lógica de negócios existentes entre as aplicações, através de um barramento  orientado a serviço, 
para que possa auxiliar o CPD na modernização dos sistemas legados da Instituição, que estão em uso há mais de 20 anos.
\end{abstract}
